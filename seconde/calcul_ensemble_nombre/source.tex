\documentclass[a4paper]{article}

\usepackage[utf8x]{inputenc}
\usepackage[T1]{fontenc}
\usepackage{amsmath}
\usepackage{amssymb}
\usepackage{fp}
%\usepackage[french]{babel}
%\usepackage[babel=true,kerning=true]{microtype}

\usepackage[francais,bloc,completemulti]{automultiplechoice}
\usepackage{csvsimple}
\usepackage{siunitx}
\sisetup{locale = FR,detect-all,quotient-mode=fraction,
input-product=*, list-final-separator = { \translate{et} },
list-pair-separator = { \translate{et} },range-phrase = { \translate{à} },
separate-uncertainty = true,group-minimum-digits=3}

\def\barQ{b=1,m=-0.5}

\newcommand{\sujet}{%
\exemplaire{1}{%
%\shorthandon{;}
%%% debut de l’en-tête des copies :
\AMCassociation{\id}

\vspace*{.5cm}
\begin{minipage}{.4\linewidth}
\centering\large\bf QCM Mathématiques 216\\ Examen du 26/09/2019
\end{minipage}
\champnom{\fbox{\begin{minipage}{.5\linewidth}

\centering\textsc{\nom{} \prenom{}}
%\vspace*{.5cm}\dotfill
\vspace*{2mm}
\end{minipage}}}

%%% fin de l’en-tête

\begin{center}
\hrule\vspace{2mm}
\bf\Large Calculs
\vspace{1mm}\hrule
\end{center}

\restituegroupe{calculs}


\begin{center}
\hrule\vspace{2mm}
\bf\Large Ensembles
\vspace{2mm}\hrule
\end{center}

\restituegroupe{ensembles}

}%fin commande \exemplaire
}%fin commande \newcommand
\begin{document}

%%% préparation des groupes

\setdefaultgroupmode{withreplacement}

\element{calculs}{
\begin{question}{odg}\bareme{\barQ}
\FPeval\VQa{trunc(3+random*5,1)}
\FPeval\VQc{clip(100-VQa*7.125)}
\FPeval\VQd{clip((100-VQa*7.125)/10)}
\FPeval\VQe{clip((100-VQa*7.125)*10)}
Sans utiliser de calculette, indique la réponse la plus probable pour le calcul de $A = 100 - \VQa{} \times 7,125$.
\begin{reponseshoriz}
\bonne{A = \VQc}
\mauvaise{\VQd}
\mauvaise{\VQe}
\end{reponseshoriz}
\end{question}
}

\element{calculs}{
\begin{question}{frac}

\FPeval\VQb{trunc(4+random*4,0)}
\FPeval\VQsomme{clip(7-6/(3-VQb))}
\FPeval\VQsommeA{clip(1/(3-VQb))}
\FPeval\VQsommeB{clip(7+6/(3-VQb))}
\FPeval\VQsommeC{clip(7+6/(3+VQb))}
Calculer $B=7 - \dfrac{6}{3-\VQb{}}$. Quelle est la bonne réponse~?
\begin{reponseshoriz}
\bonne{\VQsomme}
\mauvaise{\VQsommeA}
\mauvaise{\VQsommeB}
\mauvaise{\VQsommeC}
\end{reponseshoriz}
\end{question}
}

\element{calculs}{
\begin{question}{calc4}
L'une des expressions suivantes est égale à l'expression 2a - b. Laquelle~?
\begin{reponseshoriz}
\mauvaise{$a + (-b - a)$}
\mauvaise{$2(a - b)$}
\bonne{$a - (b - a)$}
\mauvaise{$-(b - a) - a$}
\end{reponseshoriz}
\end{question}
}


\element{calculs}{
\begin{question}{rais}
Une seule des affirmations suivantes est exacte. Laquelle~?
\begin{reponses}
\bonne{L'opposé d'une somme est la somme des opposés.}
\mauvaise{L'opposé d'un produit est le produit des opposés.}
\mauvaise{Le carré d'une somme est la somme des carrés.}
\end{reponses}
\end{question}
}

\element{calculs}{
\begin{question}{diffcarres}
Quelle est l'expression qui n'est pas une différence de deux carrés~?
\begin{reponseshoriz}
\mauvaise{$a^{2}-b^{2}$}
\mauvaise{$(a+b)^{2}-c^{2}$}
\mauvaise{$1-(a+b-c)^2$}
\bonne{$(a-b)^{2}$}
\end{reponseshoriz}
\end{question}
}

\element{ensembles}{
\begin{question}{ensemblez}
Parmi les inclusions suivantes, une seule est fausse : laquelle ?
\begin{reponseshoriz}
\bonne{$\mathbb{R}\subset\mathbb{Q}$}
\mauvaise{$\mathbb{N}\subset\mathbb{R}$}
\mauvaise{$\mathbb{Z}\subset\mathbb{Q}$}
\mauvaise{$\mathbb{Z}\subset\mathbb{D}$}
\end{reponseshoriz}
\end{question}
}


\element{ensembles}{
\begin{question}{ensemble2}
L’ensemble des nombres entiers relatifs se note :
\begin{reponseshoriz}
\mauvaise{$\mathbb{N}$}
\bonne{$\mathbb{Z}$}
\mauvaise{$\mathbb{D}$}
\mauvaise{$\mathbb{Q}$}
\mauvaise{$\mathbb{R}$}
\end{reponseshoriz}
\end{question}
}

\element{ensembles}{
\begin{questionmult}{intervalle}
\AMCnoCompleteMulti
\FPeval{\x}{trunc(1+random*9,0)}
\FPiflt{\x}{5} \FPeval\VQa{trunc(10+random*30,0)} \else \FPeval\VQa{trunc(10-random*30,0)} \fi

\FPeval\VQb{trunc(60+random*30,0)}
Si $I = [\VQa{};+\infty[$ et $J = ]-\infty;\VQb{}]$
\begin{reponseshoriz}
\bonne{$I \cap J=[\VQa;\VQb]$}\bareme{b=0.5}
\bonne{$I \cup J=]-\infty;+\infty[$}\bareme{b=0.5}
\mauvaise{$I \cup J=[\VQa;\VQb[$}\bareme{b=0,m=-0.5}
\mauvaise{$I \cap J=[-\infty;\VQa]$}\bareme{b=0,m=-0.5}
\mauvaise{$I \cup J=[-\infty;\VQb]$}\bareme{b=0,m=-0.5}

\end{reponseshoriz}
\end{questionmult}
}

\element{ensembles}{
\begin{question}{decimal}
\prenom effectue la division de 17 par 125 avec sa calculette. Comment doit-il inscrire le résultat obtenu ?
\begin{reponseshoriz}
\bonne{$\dfrac{17}{125}=0,136$}
\mauvaise{$\dfrac{17}{125}=0,136...$}
\mauvaise{$\dfrac{17}{125}\approx 0,136$}
\mauvaise{Laisser $\dfrac{17}{125}$}
\end{reponseshoriz}
\end{question}
}



\element{ensembles}{
\begin{question}{inegalite}
\FPeval{\x}{trunc(1+random*9,0)}
\FPiflt{\x}{5} \FPeval\VQa{trunc(10+random*30,0)} \else \FPeval\VQa{trunc(10-random*30,0)} \fi
\FPeval\VQb{trunc(10+random*30,0)}
\FPeval\VQb{trunc(60+random*30,0)}
$\VQa{}<x\leq\VQb{}$ est équivalent à:
\begin{reponseshoriz}
\bonne{$x\in]\VQa;\VQb]$}
\mauvaise{$x\in[\VQa;\VQb]$}
\mauvaise{$x\in]\VQa;\VQb[$}
\mauvaise{$x\in[\VQa;\VQb[$}


\end{reponseshoriz}
\end{question}
}


%%génération des sujets
\csvreader[head to column names]{liste.csv}{}{\sujet}
%\shorthandoff{;} 5 [3]
%\csvreader[head to column names,separator=semicolon]{liste.csv}{}{\sujet}
\end{document}
