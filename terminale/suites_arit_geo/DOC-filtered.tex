%%AMC:preprocess_command=prePythonTex4AMC
%%AMC:jobspecific=1
%%AMC:latex_engine=pdflatex --shell-escape

% Compilation Texmaker : Utilisateur/Commandes utilisateur
%--------------------------
% Shell Python :
% pdflatex --shell-escape -synctex=1 -interaction=nonstopmode %.tex|
% python /usr/share/texlive/texmf-dist/scripts/pythontex/pythontex.py %.tex|
% pdflatex --shell-escape -synctex=1 -interaction=nonstopmode %.tex|
%------------------------
% Copier-coller la ligne ci-dessous dans Utilisateur/Commandes Utilisateur/Editer Commandes Utilisateur
%------------------------
% pdflatex --shell-escape -synctex=1 -interaction=nonstopmode %.tex | python /usr/share/texlive/texmf-dist/scripts/pythontex/pythontex.py %.tex | pdflatex --shell-escape -synctex=1 -interaction=nonstopmode %.tex
%--------------------------

\documentclass[a4paper]{article}
\RequirePackage{etex}
\usepackage[utf8]{inputenc}    
\usepackage[T1]{fontenc}
\usepackage[french]{babel}
\selectlanguage{french}
\usepackage[francais,bloc,completemulti]{automultiplechoice}    
\usepackage{fancyhdr,amssymb,amsmath}
\usepackage{multicol}

\usepackage[onelanguage,french,linesnumbered,ruled]{algorithm2e}

% ---------- Utilisation de codes Python -----
\usepackage{pythontex}
\usepackage{csvsimple}

%----------------------------------------------
\def\barO{b=1.25,m=O}
\def\barQ{b=1,m=-0.1}
\def\barS{b=1,m=0}
\def\barP{b=1.5,m=-0.1}
\def\barR{b=2,m=-0.1}



\begin{pycode}
import numpy as np
import math
from fractions import Fraction as F
from sympy import *


np.random.seed(12345)
# IMPORTANT : Graine pour la fonction random


def polyv2(b,p,d,q) :
	x=Symbol('x')
	myp = expand((x-b)**p*(x-d)**q)
	return latex(myp)
	
def dp1x(a,b,c) :
    x=Symbol('x')
    y=Symbol('y')
    f=(a*y-1)/(b*x-1)**c

    g=diff(f,x,1)
    g2=g/b
    g3=g*(-1)
    g4=g2*(-1)

	 
    gl = latex(g)
    g2l = latex(g2)
    g3l = latex(g3)
    g4l = latex(g4)

    return np.array([gl,g2l,g3l,g4l])

def fonction(a,x1,x2) :
    x=Symbol('x')
    f=a*x*x-a*(x1+x2)*x+a*x1*x2
    fl=latex(f)
    d=np.random.randint(2,5)
    y=-a*(x1+x2)*x+a*(x1*x2+d**2)
    yl=latex(y)
    yi=a*(d**2+x1*x2-(x1+x2)*d)
    yib=a*(d**2+x1*x2+(x1+x2)*d)
    return np.array([fl,yl,d,yi,yib])

def canonique(a,b,c) :
    x=Symbol('x')
    delta=b*b-4*a*c
    xs=F(-b,2*a)
    ys=F(-delta,4*a)
    f=a*(x-xs)**2+ys
    fl = latex(f)
    f2=a*(x+xs)**2+ys
    f2l = latex(f2)
    f3=a*(x-xs)**2-ys
    f3l = latex(f3)
    f4=a*(x+xs)**2-ys
    f4l = latex(f4)
    return np.array([fl,f2l,f3l,f4l])

def sommet(a,b,c) :
    delta=b*b-4*a*c
    xs=F(b+c,2)
    ys=F(a*(xs-b)*(xs-c),1)
    return np.array([xs,ys])

def factor(b,c) :
    x=Symbol('x')
    ff=(x-b)*(x-c)
    ffl = latex(ff)
    ff2=(x-b)*(x+c)
    ff2l=latex(ff2)
    ff3=(x+b)*(x-c)
    ff3l=latex(ff3)
    ff4=(x+b)*(x+c)
    ff4l=latex(ff4)
    return np.array([ffl,ff2l,ff3l,ff4l])

def coeff() :
    pair=[2,4,6,8,10]
    impair=[1,3,5,7,9]
    a=np.random.randint(-7,7)
    while a == 0:
        a=np.random.randint(-7,7)
    x1=np.random.randint(-10,-2)
    if (x1&1) :
        x2=np.random.choice(impair) 
    else :
        x2=np.random.choice(pair)
    if x2==-x1:
        x2=x2+2
    if -x1 > x2:
        x3=x2
        x4=-x1
    else:
        x3=-x1
        x4=x2
    return np.array([a,x1,x2,x3,x4])

def image(a,b,c,x) :
    return a*x*x+b*x+c

def signe(a):
    if a > 0:
        return ("positif","negatif")
    return ("negatif","positif")

def position(a):
    if a > 0:
        return ("au-dessous","au-dessus")
    return ("au-dessus","au-dessous")

def algosuite(u0,r,q,n):
    u=u0
    for i in range(1,n+1):
        u=q*u
        u=u+r
    return u

def suitearith(a0,r,q):
    aq=a0
    for i in range(1,q+1):
        aq=r+aq
    return aq

def suitegeo(a0,r,q):
    aq=a0
    for i in range(1,q+1):
        aq=r*aq
    return aq

def sommegeo(q,n):
    return round((1-q**n)/(1-q),2)

def sommearith(u0,r,n):
    return round((u0+suitearith(u0,r,n))/2,2)

def limit(a,b,q):
    if q<1:
       return [a,0,"-\infty","+\infty"]
    else:
        if b > 0:
            return ["+\infty",a,0,"-\infty"]
    return ["-\infty",a,0,"+\infty"]

\end{pycode}




%--------
% DOCUMENT
%--------
\begin{document}

\newcommand{\sujet}{
\exemplaire{1}{
%\pyc{a=np.random.randint(2,7)} %

%\pyc{x1=np.random.randint(-6,-1)} %
 
%\pyc{x2=np.random.randint(2,8)} % 


\AMCassociation{\id}
%%% debut de l'en-tête des copies :

%%% début de l’en-tête de la feuille de réponses
\begin{minipage}{.4\linewidth}
\centering\large\bf Evaluation  Mathématiques TES4\\Vendredi 20.12.2019 \end{minipage}
\champnom{\fbox{    
                \begin{minipage}{.5\linewidth}
                  \begin{center} \nom{} \prenom{} \end{center}
		
                \end{minipage}
         }}


	%%% codage des numéros d'étudiants sur 3 chiffres par les candidats et encadré d'écriture manuelle des noms
	%\vspace*{3mm}
	%{\setlength{\parindent}{0pt}\AMCcodeGridInt{etu}{3}\hspace*{\fill}
	%\begin{minipage}[b]{12.8cm} %12cm

%----------------------------------------

	 %\hfill\champnom{\fbox{
	 %\begin{minipage}{0.96\linewidth}
	%	\vspace*{3mm} %3mm
		
	%	$\leftarrow$ {\footnotesize  Codez votre QCM-Number.\hspace{1.6ex}} || NOM :
		
	%	{\footnotesize  ( centaines, dizaines et unités )\hspace{2.19ex}} || Prénom :
		
	%	{\footnotesize  Complétez le cadre ci-contre} $\rightarrow$ \hspace{0.2ex} || Groupe :		
		%\hspace{4.438cm}|| Groupe :
	%	\vspace*{3mm}
	 %\end{minipage}
	  %  }
	  % }
	%\hfill  %\vspace{0.25ex}
%-------------------------------------


	
	
\vspace{2ex}

%%% fin de l'en-tête

Un volume constant de $2200 m^3$ d’eau est réparti entre deux bassins A et B. Le bassin A refroidit une machine.
Pour des raisons d’équilibre thermique on crée un courant d’eau entre les deux bassins à l’aide de pompes.\vspace{1ex}

On modélise les échanges entre les deux bassins de la facon suivante :
\begin{itemize}
   \item au départ, le bassin A contient $800 m^3$ d’eau et le bassin B contient $1400 m^3$ d’eau;
   \item tous les jours, $15\%$ du volume d’eau présent dans le bassin B au début de la journée est transferé vers le bassin A;
   \item tous les jours, $10\%$ du volume d’eau présent dans le bassin A au début de la journée est transféré vers le bassin B.\vspace{1ex}
\end{itemize}

Pour tout entier naturel $n$, on note :
\begin{itemize}
   \item $a_{n}$ le volume d’eau, exprimé en $m^3$, contenu dans le bassin A à la fin du $n$-ième jour de fonctionnement ;
   \item $b_{n}$ le volume d’eau, exprimé en $m^3$, contenu dans le bassin B à la fin du $n$-ième jour de fonctionnement.\vspace{1ex}
\end{itemize}

On a donc $a_0 = 800$ et $b_0 = 1400$.


\vspace{2ex}

\begin{question}{conservation}\bareme{\barQ}
Par quelle relation entre $a_n$ et $b_n$ traduit-on la conservation du volume total d’eau du circuit ?
\AMCOpen{lines=2,dots=false,scan=false}{\wrongchoice[F]{f}\scoring{0}\wrongchoice[P]{p}\scoring{0.5}\correctchoice[J]{j}\scoring{1}}
\end{question}

\begin{question}{an}\bareme{\barQ}
Justifier que, pour tout entier naturel $n$, $a_{n+1} = 0, 75 \times a_n + 330$.
\AMCOpen{lines=4,dots=false,scan=false}{\wrongchoice[F]{f}\scoring{0}\wrongchoice[P]{p}\scoring{0.5}\correctchoice[J]{j}\scoring{1}}
\end{question}


\begin{algorithm}
    \SetKwFor{Tq}{Tant que}{faire}{Fin Tant que}
    \SetKwInOut{Input}{Variables}
    \SetKwInOut{Output}{Sortie}
    \SetKwInOut{Afficher}{Afficher}
    \SetAlgoLined
    \Input{$n$ est un nombre entier naturel}
    \Input{$a$ est un nombre réèl}
    $n$ prend la valeur $0$\;
    $a$ prend la valeur $800$\;
    \Tq{a < 1100}{
	$a$ prend la valeur ...\;
	$n$ prend la valeur ...\;
	}
    \Output{Afficher $n$;}
     
\end{algorithm}

\begin{question}{algorithme}\bareme{\barQ}
L’algorithme ci-dessus permet de déterminer la plus petite valeur de $n$ à partir de laquelle $a_n$ est supérieur ou égal à 1100. Recopier et compléter les parties manquantes de cet algorithme.
\AMCOpen{lines=3,dots=false,scan=false}{\wrongchoice[F]{f}\scoring{0}\wrongchoice[P]{p}\scoring{0.5}\correctchoice[J]{j}\scoring{1}}
\end{question}


Pour tout entier naturel $n$, on note $u_n = a_n - 1320$.\\

\begin{question}{ungeo}\bareme{\barQ}
Montrer que la suite $(u_n)$ est une suite géométrique dont on précisera le premier terme et la raison.
\AMCOpen{lines=5,dots=false,scan=false}{\wrongchoice[F]{f}\scoring{0}\wrongchoice[P]{p}\scoring{0.5}\correctchoice[J]{j}\scoring{1}}
\end{question}

\begin{question}{unexplicite}\bareme{\barQ}
Exprimer $u_n$ en fonction de $n$.
\AMCOpen{lines=2,dots=false,scan=false}{\wrongchoice[F]{f}\scoring{0}\wrongchoice[P]{p}\scoring{0.5}\correctchoice[J]{j}\scoring{1}}
\end{question}

\begin{question}{anexplicite}\bareme{\barQ}
En déduire que pour tout entier naturel $n$, $a_{n}=1320-520\times0,75^n$
\AMCOpen{lines=2,dots=false,scan=false}{\wrongchoice[F]{f}\scoring{0}\wrongchoice[P]{p}\scoring{0.5}\correctchoice[J]{j}\scoring{1}}
\end{question}

\begin{question}{memevolume}\bareme{\barQ}
On cherche à savoir si, un jour donné, les deux bassins peuvent avoir, au mètre cube près, le même volume d’eau.
Proposer une méthode pour répondre à ce questionnement.
\AMCOpen{lines=4,dots=false,scan=false}{\wrongchoice[F]{f}\scoring{0}\wrongchoice[P]{p}\scoring{0.5}\correctchoice[J]{j}\scoring{1}}
\end{question}

\begin{question}{bonus}\bareme{\barQ}
BONUS: Montrer que $1+\dfrac{1}{11}+\dfrac{1}{11^2}+\dfrac{1}{11^3}+\dfrac{1}{11^4}+....=\dfrac{11}{10}$
\AMCOpen{lines=2,dots=false,scan=false}{\wrongchoice[F]{f}\scoring{0}\wrongchoice[P]{p}\scoring{0.5}\correctchoice[J]{j}\scoring{1}}
\end{question}

\AMCcleardoublepage
}


}
\csvreader[head to column names]{../liste.csv}{}{\sujet}

\end{document}
