%%AMC:preprocess_command=prePythonTex4AMC
%%AMC:jobspecific=1
%%AMC:latex_engine=pdflatex --shell-escape

% Compilation Texmaker : Utilisateur/Commandes utilisateur
%--------------------------
% Shell Python :
% pdflatex --shell-escape -synctex=1 -interaction=nonstopmode %.tex|
% python /usr/share/texlive/texmf-dist/scripts/pythontex/pythontex.py %.tex|
% pdflatex --shell-escape -synctex=1 -interaction=nonstopmode %.tex|
%------------------------
% Copier-coller la ligne ci-dessous dans Utilisateur/Commandes Utilisateur/Editer Commandes Utilisateur
%------------------------
% pdflatex --shell-escape -synctex=1 -interaction=nonstopmode %.tex | python /usr/share/texlive/texmf-dist/scripts/pythontex/pythontex.py %.tex | pdflatex --shell-escape -synctex=1 -interaction=nonstopmode %.tex
%--------------------------

\documentclass[a4paper]{article}
\RequirePackage{etex}
\usepackage[utf8]{inputenc}    
\usepackage[T1]{fontenc}

\usepackage[francais,bloc,completemulti]{automultiplechoice}    
\usepackage{fancyhdr,amssymb,amsmath}

% ---------- Utilisation de codes Python -----
\usepackage{pythontex}
\usepackage{csvsimple}
%----------------------------------------------
\def\barQ{b=1,m=-0.25}

\begin{pycode}
import numpy as np
import math
from fractions import Fraction as F
from sympy import *
from mpmath import *
from decimal import *
mp.dps = 30; mp.pretty = True


np.random.seed(12345)
# IMPORTANT : Graine pour la fonction random


def polyv2(b,p,d,q) :
	x=Symbol('x')
	myp = expand((x-b)**p*(x-d)**q)
	return latex(myp)
	
def dp1x(a,b,c) :
    x=Symbol('x')
    y=Symbol('y')
    f=(a*y-1)/(b*x-1)**c

    g=diff(f,x,1)
    g2=g/b
    g3=g*(-1)
    g4=g2*(-1)
	 
    gl = latex(g)
    g2l = latex(g2)
    g3l = latex(g3)
    g4l = latex(g4)

    return np.array([gl,g2l,g3l,g4l])

def fonction(a,x1,x2) :
    x=Symbol('x')
    f=a*x*x-a*(x1+x2)*x+a*x1*x2
    f2=a*x*x+a*(x1+x2)*x
    f3=a*x*x+a*(x1+x2)*x+a*x1*x2
    f4=3*a*x*x-a*(x1+x2)*x+a*x1*x2
    fl=latex(f)
    d=np.random.randint(2,5)
    y=-a*(x1+x2)*x+a*(x1*x2+d**2)
    yl=latex(y)
    yi=a*(d**2+x1*x2+(x1+x2)*d)
    yib=a*(d**2+x1*x2-(x1+x2)*d)
    return np.array([fl,yl,d,yi,yib,f,f2,f3,f4])

def resolution2d2(a,x1,x2) :
    x=Symbol('x')
    k=np.random.randint(-8,-1)
    f=a*x*x-a*(x1+x2)*x+a*x1*x2+k
    f2=a*x*x-a*(x1+x2)*x+a*x1*x2
    fl=latex(f)
    return np.array([fl,k,f2])

def canonique(a,b,c) :
    x=Symbol('x')
    delta=b*b-4*a*c
    xs=F(-b,2*a)
    ys=F(-delta,4*a)
    f=a*(x-xs)**2+ys
    fl = latex(f)
    f2=a*(x+xs)**2+ys
    f2l = latex(f2)
    f3=a*(x-xs)**2-ys
    f3l = latex(f3)
    f4=a*(x+xs)**2-ys
    f4l = latex(f4)
    return np.array([fl,f2l,f3l,f4l])

def sommet(a,b,c) :
    delta=b*b-4*a*c
    xs=F(b+c,2)
    ys=F(a*(xs-b)*(xs-c),1)
    return np.array([xs,ys])

def factor(b,c) :
    x=Symbol('x')
    ff=(x-b)*(x-c)
    ffl = latex(ff)
    ff2=(x-b)*(x+c)
    ff2l=latex(ff2)
    ff3=(x+b)*(x-c)
    ff3l=latex(ff3)
    ff4=(x+b)*(x+c)
    ff4l=latex(ff4)
    return np.array([ffl,ff2l,ff3l,ff4l])

def coeff() :
    pair=[2,4,6,8,10]
    impair=[1,3,5,7,9]
    a=np.random.randint(-7,7)
    while a == 0:
	a=np.random.randint(-7,7)
    x1=np.random.randint(-10,-2)
    if (x1&1) :
    	x2=np.random.choice(impair) 
    else :
	x2=np.random.choice(pair)
    if x2==-x1:
        x2=x2+2
    if -x1 > x2:
	x3=x2
	x4=-x1
    else:
	x3=-x1
	x4=x2
    return np.array([a,x1,x2,x3,x4])

def image(a,b,c,x) :
    return a*x*x+b*x+c

def signe(a):
    if a > 0:
        return ("positif","negatif")
    return ("negatif","positif")

def position(a):
    if a > 0:
        return ("au-dessous","au-dessus")
    return ("au-dessus","au-dessous")

def primitive(f):
	x=Symbol('x')
	k=np.random.randint(-7,7)	
	while k == 0:
		k=np.random.randint(-7,7)
	F=integrate(f)+k
	return F

def deriveepoly(a,b,c):
	x=Symbol('x')
	d=2*a*x+b
        return latex(d)

def tangentepoly(a,b,c):
	k=np.random.randint(-7,7)	
	while k == 0:
		k=np.random.randint(-7,7)
        x=Symbol('x')
        t=(2*a*k+b)*(x-k)+a*k**2+b*k+c
	t2=(2*a*k+b)*(x+k)+a*k**2+b*k+c
	t3=(2*a*k+b)*(x+k)-a*k**2-b*k-c
	t4=(2*a*k+b)*(x-k)-a*k**2-b*k-c
        tl=latex(t)
	t2l=latex(t2)
	t3l=latex(t3)
	t4l=latex(t4)
	return np.array([k,tl,t2l,t3l,t4l])

def variation(a):
	if a>0:
		return ("croissante","decroissante")
	return ("decroissante","croissante")

def image(f,a):
	x=Symbol('x')
	return sympify(f).subs(x,a)
	
def solution(f,x2f):
	x=Symbol('x')
	P = Poly(f,x)
	co = P.all_coeffs()
	return np.roots(co)

def encadrement(a,x1,x2,sols):
	if sols[0]>=sols[1]:
		if sols[0]>=sols[2]:
			alpha=sols[0]
			if sols[1]>=sols[2]:
				minus=sols[2]
			else:
				minus=sols[1]
		else:
			alpha=sols[2]
			minus=sols[1]
	else:
		if sols[1]>=sols[2]:
			alpha=sols[1]
			if sols[0]>=sols[2]:
				minus=sols[2]
			else:
				minus=sols[0]
		else:
			alpha=sols[2]
			minus=sols[0]
	if alpha <= x1:
		xmax=(alpha+x1)/2
		xmin=2*alpha-xmin
	else:
		xmin=(alpha+x2)/2
		xmax=2*alpha-xmin
			
	return np.array([xmin,xmax,alpha,minus])

def solutionI(sols,x1,x2):
	if ((sols[0] > x1) and (sols[0] < x2)) or ((sols[1] > x1) and (sols[1] < x2)) or ((sols[2] > x1) and (sols[2] < x2)):
		return ("admet une unique","nadmet pas de")
	return ("n'admet pas de","admet une unique")


def roundD(a,n):
	d=Decimal(a)
	if n == 0:
		return Decimal(d.quantize(Decimal('?01'), rounding=ROUND_HALF_DOWN))
	return Decimal(d.quantize(Decimal('?01'), rounding=ROUND_HALF_UP))

\end{pycode}

%--------
% DOCUMENT
%--------
\begin{document}

\newcommand{\sujet}{
\exemplaire{1}{
%\pyc{a=np.random.randint(2,7)} %

%\pyc{x1=np.random.randint(-6,-1)} %
 
%\pyc{x2=np.random.randint(2,8)} % 


\AMCassociation{\id}
%%% debut de l'en-tête des copies :    

%\noindent{\bf QCM  \hfill TES4}

%\vspace*{.5cm}
\begin{minipage}{.4\linewidth}
\centering\large\bf Devoir Mathématiques 1\\ Mardi 03.10.2019 \end{minipage}
\champnom{\fbox{    
                \begin{minipage}{.5\linewidth}
                 \begin{center} \nom{} \prenom{} \end{center}

                \end{minipage}
         }}


	%%% codage des numéros d'étudiants sur 3 chiffres par les candidats et encadré d'écriture manuelle des noms
	%\vspace*{3mm}
	%{\setlength{\parindent}{0pt}\AMCcodeGridInt{etu}{3}\hspace*{\fill}
	%\begin{minipage}[b]{12.8cm} %12cm

%----------------------------------------

	 %\hfill\champnom{\fbox{
	 %\begin{minipage}{0.96\linewidth}
	%	\vspace*{3mm} %3mm
		
	%	$\leftarrow$ {\footnotesize  Codez votre QCM-Number.\hspace{1.6ex}} || NOM :
		
	%	{\footnotesize  ( centaines, dizaines et unités )\hspace{2.19ex}} || Prénom :
		
	%	{\footnotesize  Complétez le cadre ci-contre} $\rightarrow$ \hspace{0.2ex} || Groupe :		
		%\hspace{4.438cm}|| Groupe :
	%	\vspace*{3mm}
	 %\end{minipage}
	  %  }
	  % }
	%\hfill  %\vspace{0.25ex}
%-------------------------------------

\begin{minipage}{0.96\linewidth}
\vspace*{3mm} %3mm
% texte de presentation 
%TEST125;Jean125;125 ## TEST102;Pierre102;102 TEST114;Marie114;114 ##TEST269;Helen269;269
\begin{center}\em  %\bf {center}

Les questions faisant apparaître le symbole \multiSymbole{} présentent  plusieurs bonnes réponses.
Les questions ont une unique bonne réponse. L'indiquer sur cette feuille en noircissant la case correspondante au stylo à bille noir. Aucune justification n'est demandée.
Les réponses fausses retirent la moitié des points. Une absence de réponse n'enlève pas de points.
Pour rectifier une erreur, utilisez un correcteur "blanc" pour faire disparaître complètement la case noircie par erreur.	Calculatrice autorisée.
 

\end{center}

	\end{minipage}\hspace*{\fill}
	
	
\vspace{5ex}

%%% fin de l'en-tête
\pyc{coefs=coeff()}
\pyc{af=coefs[0]}
\pyc{x1f=coefs[1]}
\pyc{x2f=coefs[2]}
\pyc{x3f=coefs[3]}
\pyc{x4f=coefs[4]}
\pyc{foncf=fonction(af,x1f,x2f)}

\begin{question}{1resolutionf0}\bareme{\barQ}
Soit la fonction  $f(x)= \py{foncf[0]}$. Résoudre $f(x)=0$
%On donne la fonction suivante : $f(x)= \py{fonc[0]}$ 
%\pyc{canon=canonique(a,-a*(x1+x2),a*x1*x2)}	

			\begin{reponseshoriz}
				\bonne   {$S=\left\{\py{x1f};\py{x2f}\right\}$}
				\mauvaise{$S=\left\{\py{x3f};\py{x4f}\right\}$}
                                \mauvaise{$S=\left\{\py{-x4f};\py{-x3f}\right\}$}
                                \mauvaise{$S=\left\{\py{-x2f};\py{-x1f}\right\}$}
			\end{reponseshoriz}
\end{question}


\begin{questionmult}{2signef}\bareme{\barQ}
\AMCnoCompleteMulti
Quel est le signe de f(x)?

			\begin{reponseshoriz}
				\bonne    {Il est \pyc{print(signe(af)[0])}à l'exterieur des racines.}
				\mauvaise {Il est \pyc{print(signe(af)[0])}à l'interieur des racines.}
				\bonne {Il est \pyc{print(signe(af)[1])}à l'interieur des racines.}
				\mauvaise {Il est \pyc{print(signe(af)[1])}à l'exterieur des racines.}
			\end{reponseshoriz}
\end{questionmult}


\pyc{coefs=coeff()}
\pyc{a=coefs[0]}
\pyc{x1=coefs[1]}
\pyc{x2=coefs[2]}
\pyc{x3=coefs[3]}
\pyc{x4=coefs[4]}
\pyc{foncg=resolution2d2(a,x1,x2)}
\pyc{k=foncg[1]}

\begin{question}{3resolutiongk}\bareme{\barQ}
Soit la fonction  $g(x)= \py{foncg[0]}$. Résoudre $g(x)=\py{k}$

			\begin{reponseshoriz}
				\bonne   {$S=\left\{\py{x1};\py{x2}\right\}$}
				\mauvaise{$S=\left\{\py{x3};\py{x4}\right\}$}
                                \mauvaise{$S=\left\{\py{-x4};\py{-x3}\right\}$}
                                \mauvaise{$S=\left\{\py{-x2};\py{-x1}\right\}$}
			\end{reponseshoriz}
\end{question}


\pyc{coefs=coeff()}
\pyc{a=coefs[0]}
\pyc{x1=coefs[1]}
\pyc{x2=coefs[2]}
\pyc{x3=coefs[3]}
\pyc{x4=coefs[4]}
\pyc{fonch=fonction(a,x1,x2)}
\pyc{tangh=tangentepoly(a,-a*(x1+x2),a*x1*x2)}

\begin{question}{4tangente}\bareme{\barQ}
Quelle est l'équation de la tangente à la courbe représentative de la fonction $h(x)=\py{fonch[0]}$ au point d'abscisse \py{tangh[0]} ?

			\begin{reponseshoriz}
				\bonne   {$y=\py{tangh[1]}$}
				\mauvaise{$y=\py{tangh[2]}$}
                                \mauvaise{$y=\py{tangh[3]}$}
				\mauvaise{$y=\py{tangh[4]}$}
			\end{reponseshoriz}
\end{question}



%\pyc{fonciprime=fonction(a,x1,x2)}
\pyc{fonci=primitive(foncf[5])}

\begin{question}{5derivee}\bareme{\barQ}
Quelle est la dérivée de la fonction $i(x)=\py{latex(fonci)}$ ?

			\begin{reponseshoriz}
				\bonne   {$i'(x)=\py{foncf[0]}$}
				\mauvaise{$i'(x)=\py{latex(foncf[6])}$}
                                \mauvaise{$i'(x)=\py{latex(foncf[7])}$}
				\mauvaise{$i'(x)=\py{latex(foncf[8])}$}
			\end{reponseshoriz}
\end{question}

\begin{question}{6variations}\bareme{\barQ}
\pyc{var=variation(af)}
En déduire le sens de variation de la fonction i ?

			\begin{reponseshoriz}
				\bonne   {$i$ est \pyc{print(var[0])}sur $]-\infty;\py{x1f}]\cup[\py{x2f};+\infty[$ et \pyc{print(var[1])}sur $[\py{x1f};\py{x2f}]$}
				\mauvaise{$i$ est \pyc{print(var[1])}sur $]-\infty;\py{x1f}]\cup[\py{x2f};+\infty[$ et \pyc{print(var[0])}sur $[\py{x1f};\py{x2f}]$}
                                \mauvaise{$i$ est \pyc{print(var[0])}sur $]-\infty;\py{x3f}]\cup[\py{x4f};+\infty[$ et \pyc{print(var[1])}sur $[\py{x3f};\py{x4f}]$}
				\mauvaise{$i$ est \pyc{print(var[1])}sur $]-\infty;\py{x3f}]\cup[\py{x4f};+\infty[$ et \pyc{print(var[0])}sur $[\py{x3f};\py{x4f}]$}
			\end{reponseshoriz}
\end{question}


\pyc{sols=solution(fonci,x2f)}
%Nombre de solutions de $i(x)=0$ est $\py{len(sols)}$
%Solutions de $i(x)=0$ sont $\py{sols[0]}$  et $\py{sols[1]}$ et $\py{sols[2]}$ \\
\pyc{encad=encadrement(af,x1f,x2f,sols)}
\pyc{solI=solutionI(sols,x1f,x2f)}
\begin{questionmult}{7existence}\bareme{\barQ}
\AMCnoCompleteMulti
On souhaite vérifier si l'équation $i(x)=0$ admet une solution sur l'intervalle $I=[x1;x2]$ avec $x1$ et $x2$ les abscisses des extremums de $i$. Cocher les affirmations exactes.
%Quelles conditions permettent d'utiliser le théorème des valeurs intermédiaires ? L'équation admet-elle une solution?

			\begin{reponseshoriz}
				\bonne   {i est continue sur l'intervalle I.}
				\bonne   {i est strictement monotone sur l'intervalle I.}
                                \mauvaise{i n'est pas monotone sur l'intervalle I.}
				\bonne{$i(x)=0$ \pyc{print(solI[0])}solution sur I.}
				\mauvaise{$i(x)=0$ \pyc{print(solI[1])}solution sur I.}
			\end{reponseshoriz}
\end{questionmult}

%$i(\py{x2f})=\py{latex(image(fonci,x2f))}$
%$i(\py{x2f+1})=\py{latex(image(fonci,x2f+1))}$
%\pyc{ialpha=(image(fonci,x2f)+image(fonci,x2f+1))/2}


\begin{question}{8encadrement}\bareme{\barQ}
Le Théorème des Valeurs Intermédiaires permet d'affirmer qu'il existe une solution $\alpha>x2$ pour l'équation $i(x)=0$. A l'aide de la calculatrice, trouver une valeur approchée de $\alpha$ au centième.


			\begin{reponseshoriz}
				\mauvaise   {$\alpha\approx\py{round(encad[1],2)}$}
				\bonne{$\alpha\approx\py{round(encad[2],2)}$}
                                \mauvaise{$\alpha\approx\py{round(encad[0],2)}$}
				\mauvaise{$\alpha\approx\py{round(encad[2],1)}$}
				\mauvaise{$\alpha\approx\py{round(encad[3],2)}$}
			\end{reponseshoriz}
\end{question}

%---------------
%--

}
}
\csvreader[head to column names]{liste.csv}{}{\sujet}

\end{document}
